% interface=en

%
% This document describes xplain2sql version 2.0 or higher
%

% complete chapter on how Xplain can support legacy systems.

\setuppapersize
  [A4][A4]

\setupbodyfont
  [postscript,11pt]

\setuplayout
  [backspace=3.5cm,width=13.8cm]

\setupwhitespace
  [medium]

\setuphead
  [chapter]
  [style={\ssd\bfd}]

\setuphead
  [section]
  [style={\ssb\bfb}]

\setuphead
  [subsection]
  [style={\ssa\bfa}]

% allow a bit more room for hyphentation
\setuptolerance
  [horizontal,strict]

%\adaptlayout[2][height=+2cm]

\setupitemize
  [packed]


% URLs

\useURL
  [bekkehome]
  [http://mmdb.kbs.twi.tudelft.nl/terBekke.html]

\useURL
  [myxplainpage]
  [http://www.pobox.com/~berend/xplain/]

\useURL
  [myxplain2sqlpage]
  [http://www.pobox.com/~berend/xplain2sql/]

\useURL
  [myeiffelpage]
  [http://www.pobox.com/~berend/eiffel/]
  []
  [my Eiffel page]

\useURL
  [eiffelforum]
  [http://www.eiffel-forum.org]

\useURL
  [Xplainmailinglist]
  [http://groups.yahoo.com/group/Xplain]

\useURL
  [eiffelstruggle1999]
  [http://www.eiffel-nice.org/eiffelstruggle/1999/struggle1999.html]
  []
  [Eiffel Struggle 1999]

\useURL
  [berend]
  [mailto:berend@pobox.com]
  []
  [berend@pobox.com]

\useURL
  [mysqlgotchas]
  [http://sql-info.de/mysql/gotchas.html]
  []
  [MySQL Gotchas]

\useURL
  [mysql-temporary-table-problems]
  [http://dev.mysql.com/doc/refman/5.0/en/temporary-table-problems.html]
  []
  [Temporary Table Problems]

\setupcolors
  [state=start]

\setupinteraction
	[state=start,
	 title={xplain2sql 4.1.0 manual},
	 author={Berend de Boer},
	 subtitle={There is more than SQL},
	 keywords={Xplain SQL conversion xplain2sql},
	 color=black,
	 menu=on]

\setuppagenumbering
  [location=]

\setupheadertexts
  [section]
  [pagenumber]

\setuptyping
  [margin=2em]


\unprotect
% \installprettytype
%   [XPLAIN]
%   [XPLAIN]
\definetyping
  [XPLAIN]
  [option=XPLAIN]
\definetyping
  [XPLAIN]
  [option=XPLAIN]
\definepalet [XPLAINcolorpretty] [colorpretty]
\definepalet [XPLAINgraypretty]  [graypretty]
\protect

%\input verb-xpl.tex
\setuptyping
  [XPLAIN]
  [margin=2em,
   style=\rm,
   palet=,icommand=\it,vcommand=,ccommand=]

% Xplain keywords
\def\Xplainkeyword#1{\index{{\it #1} command}{\it #1}}
\def\Xplainfunction#1{\index{{\it #1} function}{\it #1}}

\definetyping
  [SQL]
  [option=SQL]

\setuptyping
  [SQL]
  [margin=2em,
   palet=,icommand=\bf,vcommand=,ccommand=\it]

% XML

\definetyping
  [XML]
  [option=XML]

\setuptyping[XML]
  [margin=1em,
   before={\blank[medium]},
   after={\blank[medium]},
   style=\ss\tfx]

% default is no XML line breaks
\def\disableXMLlinebreaks{\let\XMLlinebreak\relax}
\def\enableXMLlinebreaks{\gdef\XMLlinebreak{\ifhmode\allowbreak\fi}}

%\disableXMLlinebreaks

% typeset tags
\def\XMLtag{%
  \dosingleempty\doXMLtag}

\def\doXMLtag[#1]#2{{%
  \setuptype[style=\ss]
  \doifemptyelse{#1}%
     {\color[colorprettythree]{\type{<#2>}}}%
     {\color[colorprettythree]{\type{<}}\color[colorprettyone]{\type{#1}}\color[colorprettytwo]{:}\color[colorprettythree]{\type{#2}}\color[colorprettythree]{\type{>}}}
}}


% Xplain datamodels
%\useexternalfigure[classification][classification.1][method=mpost]
%\useexternalfigure[aggregation][aggregation.1]
%\useexternalfigure[generalization][generalization.1]
%\useexternalfigure[employee][employee.1]
%\useexternalfigure[bank][bank.1]

% object models
%\useexternalfigure[sqlgenerator][sql_generator][method=png,width=\textwidth]
%\useexternalfigure[xplainexpression][xplain_expression][method=png,width=\textwidth]
%\useexternalfigure[xplainobject][xplain_object][method=png,width=\textwidth]

% SQL generation
%\input verb-sql.tex

\usemodule[bib]
\usemodule[bibltx]

\setupbibtex
  [database=/home/berend/context/tex/texmf-local/bibtex/berend,
   sort=author]

\setuppublications
  [numbering=no,
   sort=author]

% use flowcharts
\usemodule[chart]

\setupFLOWcharts
  [dx=\bodyfontsize,
  dy=\bodyfontsize,
  width=6\bodyfontsize,
  height=4.5\bodyfontsize]


\startFLOWchart[flow]

\setupFLOWcharts
[nx=6,
ny=1]

\startFLOWcell
\name {ddl}
\location {1,1}
\shape {singledocument}
\text {{\tt file.ddl}}
\connection [rl] {tokenizer}
\stopFLOWcell

\startFLOWcell
\name {tokenizer}
\location {2,1}
\shape {procedure}
\text {{\it tokenize input}}
\connection [rl] {parser}
\stopFLOWcell

\startFLOWcell
\name {parser}
\location {3,1}
\shape {procedure}
\text {{\it parse input}}
\connection [rl] {AST}
\stopFLOWcell

\startFLOWcell
\name {AST}
\location {4,1}
\shape {procedure}
\text {{\it build Xplain AST}}
\connection [rl] {createSQL}
\stopFLOWcell

\startFLOWcell
\name {createSQL}
\location {5,1}
\shape {procedure}
\text {{\it output SQL}}
\connection [rl] {sql}
\stopFLOWcell

\startFLOWcell
\name {sql}
\location {6,1}
\shape {singledocument}
\text {{\tt file.sql}}
\stopFLOWcell

\stopFLOWchart


% Format of Eiffel code
\def\EClass#1{\color[blue]{#1}}

\starttext

\startstandardmakeup
\hbox{
  \hbox{\vtop{\hsize 5.5cm\rightskip=0pt plus 1fill\baselineskip=24pt%
      \ss\bfd xplain2sql 4.1.0 manual\par\vskip12pt%
      \bfa 2011-December-20}}%
  \hbox{\vtop{\hrule width4pt height18pt depth5cm}}%
  \lower40pt\hbox{\vtop{\hsize 6cm\leftskip=0pt plus 1fill
	{\setupbodyfont[24pt,rm]\it Xplain Technology Ltd}\par\color[darkgray]{by Berend de Boer}\par}}%
}
\vbox{\vskip8cm}
\stopstandardmakeup


% _ should be normal character (define last, something overrides it)
\catcode`\_=12

\startfrontmatter

\completecontent


\chapter{Introduction}

This document is a manual for compiling and using
xplain2sql. xplain2sql is an Xplain to SQL converter. Xplain is a data
definition and data manipulation language, just like SQL. However,
Xplain has many advantages. It's a very clean, concise and orthogonal
language. Unlike SQL, in Xplain there's usually just one solution to a
problem, not dozens. Xplain is also easier to learn. And it supports
both aggregation and generalization. Both the {\it is-a} and {\it
has-a} using an elegant definition language. Inheritance in data models
is still something one seldom encounters. With Xplain this is natural
and easy. Xplain therefore is a perfect complement to an all
object-oriented language like Eiffel. By the way, in
chapter~\in[sec:Xplain] Xplain is treated in more detail and some
examples are given.

There's one problem however, you can draw Xplain diagrams and write
Xplain code, but there's no\footnote{There is also an Xplain
product. But compared to todays tools, this is quite primitive. It's
also a one-tier system, so therefore not accessible from modern RAD
tools like Delphi.} way to execute it. In the Xplain book
(see \cite[Bekke1992]) some guidelines to convert Xplain to SQL
are given. But wouldn't it be really useful to be able to convert
Xplain to SQL automatically?

And that was the start of a long list of tools. The first such a tool
was written in Turbo Pascal 7.0. Later versions were written in
Delphi. As they mostly only did the data definition part, they were
named ddl2sql. These converters have been used on serious business
systems, for example to generate the Microsoft SQL Server 6.5 code on
the new administrative system for Urk, Fish Auction, the largest
flat-fish auction in Europe.

When I wanted to write some serious programs on Unix, I started to
look for a language that compiled to native code, was cross-platform,
object-oriented, and not C++. I knew Eiffel a long time, having used
design-by-contract for 5 years. But I hadn't used Eiffel. I was
surprised by the activity and tools that were available at that
moment. Especially SmallEiffel (now called SmartEiffel) attracted me
as I could use familiar Emacs, a compiler that wasn't buried behind
layers of a supposedly fancy IDE. It was time to learn Eiffel, the
language. And there was the \from[eiffelstruggle1999]. It was time to
write the greatest Xplain converter of all times, which converted the
entire Xplain language to an SQL dialect of your choice.

With xplain2sql two goals are achieved:
\startitemize[n]
\item Write database code in an object-oriented data definition
  language.
\item Write it once, and port to any SQL dialect you want.
\stopitemize

So here it is, xplain2sql. This document explains how to compile and
install it, how to use it. Also it gives a short introduction to the
Xplain language itself (see \in{chapter}[sec:Xplain]). More
information can be found on my Xplain page at \from[myxplainpage].

And of course, this document has been created with Emacs and compiled
to \cap{pdf} with \ConTeXt.


\section{Acknowledgements}

It has been a gratifying experience to know that so many people use
this product for so many different purposes. So thanks to all people
who use this product and have told me something about how they use
it. Many users have also sent me suggestions for
improvement. Especially I want to thank:

\startitemize
\item Martin van Dinther for the very sensible suggestion to translate
  A to varchars instead of chars. Also thanks for your suggestions on
  improving the Oracle output.
\item Mark Hissink for trying the Microsoft Access output.
\item Jacco Eerland for suggesting a fix so the middleware code would
  compile with Delphi 7.
\item John Wiggings for suggestions to improve PostgreSQL output.
\stopitemize




\chapter{License and Support}

\section{License}
xplain2sql is an Open Source program. It's available under the MIT
License. You can find more details in the \filename{LICENSE} file,
provided with the xplain2sql distribution.

\section{Mailinglist}

You can subscribe to the Xplain mailing list. Subject of this
mailing list is Xplain the language and related tools like
xplain2sql. Go to \from[Xplainmailinglist] to read current message and
to subscribe.

\section{Paid support}

If your company is interested in Xplain or xplain2sql, you can ask me
to provide a one or two day course in Xplain and/or xplain2sql. Prices
are 1200 NZD a day, excluding VAT, travel and hotel expenses. Contact
me at \from[berend]. If your company is not satisfied about the
course, the fee is refunded (excluding the expenses).


\stopfrontmatter


\startbodymatter

\chapter{Compilation and installation}

This chapter explains how to compile and install xplain2sql. If you
want to avoid compiling xplain2sql, packages and binaries for several
platforms are available at \from[myxplain2sqlpage].


\section{Requirements}

\subsection{Supported platforms}

xplain2sql compiles on both Unix (tested on Ubuntu 11.94 and FreeBSD
8.2) and Windows (tested on Windows 2000 sp2). Any platform with a
Standard C compiler should be fine.

\subsection{Supported compilers}

I suggest you use the C code release of xplain2sql. It can be compiled
with any Standard C compiler.

If you want to compile the Eiffel sources, things are only slightly
more difficult. You can use either \index{SmartEiffel}SmartEiffel
1.2r7 or \index{ISE Eiffel}ISE Eiffel 6.8 or later. Earlier
SmartEiffel releases or the SmartEiffel 2.x fork are not supported.


\subsection{Libraries}

If you want to compile the Eiffel sources of xplain2sql, you need to
have \index{Gobo}Gobo 3.9 or later installed on your system.


\section{Compiling xplain2sql C source}

Just unzip the archive:

\starttyping
unzip xplain2sql-3.0-csrc.zip
\stoptyping

And run make:

\starttyping
make
\stoptyping

If you use the free Borland C 5.5 compiler, make sure to use the
special target:

\starttyping
make xplain2sql-bcc
\stoptyping


\section{Compiling xplain2sql Eiffel source}

If you have the Gobo tools in your path, building it on any platform
using SmartEiffel can be done with:

\starttyping
geant compile_se
\stoptyping

For ISE Eiffel it is:

\starttyping
geant compile_ise
\stoptyping

There is a \filename{Makefile}, but this is for development only.


\section{Testing the installation}

If you have the full Eiffel sources, and you use a Unix platform, you
can run a few test targets:

\starttyping
make test
\stoptyping

or

\starttyping
make bank
\stoptyping

Errors are written to \filename{test.err} in the first example or
\filename{bank.err} in the second. There should only be warnings in
\filename{test.err}. \filename{bank.err} should be empty.

On NT, you need to have some Unix utilities on your path like
\filename{rm} and \filename{cat}. If you have the NT resource Kit, the POSIX
utilities will do fine. You probably can also use the win32 port from
Cygnus. And with a decent \filename{make},
you should be able to execute these commands too. I've been not very
lucky with makes on NT. Many of the NT makes seem to have problems
with redirection inside a Makefile.



\section{Installation}

Installation is as simple as copying \filename{plain2sql} or
\filename{xplain2sql.exe} to a directory somewhere in your path. On Unix
systems this can be \filename{/usr/local/bin}, on NT this can be
\filename{/winnt/system32}.

You can also copy the man file \filename{xplain2sql.1} somewhere in
your man path.




\chapter{Using xplain2sql}

\section{Converting xplain files}
To convert an Xplain ddl to some dialect of SQL, type:

\starttyping
xplain2sql -tsql test.ddl
\stoptyping

The above example translates the Xplain \filename{test.ddl} to
Transact-SQL, writing it to standard output. \filename{xplain2sql} exits
with \index{error level}exit level 1 if a fatal error has occurred.

To generate output for MySQL, use:

\starttyping
xplain2sql -mysql test.ddl
\stoptyping

To save output to \filename{test.sql}, redirect the output as usual:

\starttyping
xplain2sql -tsql test.ddl > test.sql
\stoptyping

In table~\in[table:dialects] the SQL dialects xplain2sql supports are
given. See also section~\in[sec:limits] about known limitations.

\placetable[][table:dialects]{SQL dialects supported by xplain2sql}
\starttable[|l|p(11cm)|]
\HL
\VL Dialect \VL Remark \VL\SR
\HL
\VL -ansi
\VL Output ansi-92 sql.\VL\SR
\HL
\VL -basic
\VL Output basic sql statements. \VL\SR
\HL
\VL -db26
\VL Output DB/2 6 sql statements. \VL\SR
\HL
\VL -db2
\VL \VL\LR
\VL -db271
\VL Output DB/2 7.1 sql statements (tested against DB/2 9.5.0). \VL\SR
\HL
\VL -firebird
\VL \VL\LR
\VL -firebird21
\VL Output FireBird 2.1 sql (tested against version 2.1.0.17798). \VL\SR
\HL
\VL -interbase
\VL \VL\LR
\VL -interbase6
\VL Output InterBase or FireBird 1.x sql (tested against version 1.0.3). \VL\SR
\HL
\VL -mysql
\VL \VL\LR
\VL -mysql5
\VL Output MySQL 5.0 (tested against version 5.1.54).\VL\SR
\HL
\VL -mysql4
\VL Output MySQL 4.0 or higher. Only data definition
  could be made to work more or less.\VL\SR
\HL
\VL -mysql322
\VL Output MySQL 3.22. Only data definition
  could be made to work more or less.\VL\SR
\HL
\VL -oracle
\VL \VL\LR
\VL -oracle901
\VL Output Oracle sql (tested on Oracle 9.0.1). \VL\SR
\HL
\VL -pgsql
\VL \VL\LR
\VL -pgsql81
\VL Output PostgreSQL 8.1 sql (tested against version 8.4.10). \VL\SR
\HL
\VL -pgsql73
\VL Output PostgreSQL 7.3 sql (tested against version 7.3.3). \VL\SR
\HL
\VL -sqlite
\VL \VL\LR
\VL -sqlite3
\VL Output SQLite version 3 (tested against version 3.2.7). \VL\LR
\HL
\VL -tsql65
\VL Output Microsoft Transact-SQL 6.5. Purging of columns doesn't
work. And Microsoft SQL Server 6.5 is less stable than SQL Server 7.0,
so a statement might generate a crash of the server.\VL\LR
\HL
\VL -tsql70
\VL Output Microsoft Transact-SQL 7.0. \VL\LR
\HL
\VL -tsql
\VL \VL\LR
\VL -tsql2000
\VL Output Microsoft SQL Server 2000 Transact-SQL (tested against SQL Server Express 2005). \VL\LR
\HL
\stoptable

In table~\in[table:options] the other options xplain2sql supports are
given. Options only have affect if the selected SQL dialect supports
them.

\placetable[][table:options]{Options supported by xplain2sql}
\starttable[|l|p(11cm)|]
\HL
\VL Option \VL Explanation \VL\SR
\HL
\VL -attributenull
\VL Attributes (columns) are normally not null except BLOB
  columns. With this option columns are null by default. \VL\SR
\HL
%% \VL -inheritkey
%% \VL A specializations primary key will be equal to the primary key of
%%   its parent in case it has just one parent. This option is mainly
%%   there to support databases created with previous versions of
%%   xplain2sql. Its use is depreciated. \VL\SR
%% \HL
\VL -noauto
\VL Do not create auto-generated primary keys. Auto-generated primary
  keys are the default, so you don't have to care about primary keys in
  your applications. \VL\SR
\HL
\VL -nodatabase
\VL Do not output code for the Xplain database command. This makes it
  easier to create scripts that can be run in different
  databases. Without this option the Xplain database command will change
  to current database to the set database first.\VL\SR
\HL
\VL -nosp
\VL Don't create a stored procedure to insert rows. \VL\SR
\HL
\VL -timestamp
\VL Add the fieldname \type{ts_<type name>} of type `timestamp' to
  every table. This is a Microsoft SQL Server specific option. Note
  that the `timestamp' type has nothing todo with date or
  time. Timestamps are degenerate-degenerate field calls, but it's the
  only form of concurrency control we have with MS SQL Server.\VL\SR
\HL
\VL -view
\VL Create a view on every table. Could be useful if you want to have
  security only on views, and not on tables. \VL\LR
\HL
\VL -xml
\VL Generate an XML description of the generated SQL code. This XML
  description is very useful to generate wrapper classes for your
  favorite programming language.\VL\LR
\HL
\stoptable


\section[sec:limits]{Known limitations}

The conversion is not (yet) optimal for all supported
dialects. Sometimes Xplain is not fully supported, and sometimes
limitations in the SQL dialect or a not yet finished Builder gives
problems. This chapter also mentions some limitations in support for
xplain2sql extensions described in the next chapter.

\subsection{No full support for the Xplain language yet}

The Xplain language is supported to a large degree, but still some
things are missing. Most important are:

\startitemize[n]
\item Names are case-sensitive. This can be changed fairly easily, but
  a consistent spelling helps reading Xplain source in the author's
  opinion.
\item Data types are not checked for correctness. When a string is
  assigned to an integer, xplain2sql relies on the SQL dialect to
  convert it. And that does not always succeed.
\item The \Xplainkeyword{assert} command should work for all dialects,
  provided they support at least views, but no code is generated to
  check if the value of an assertion is in the
  specified range. For example:
\startXPLAIN
assert invoice its lines (1..*) = count invoice line per invoice.
\stopXPLAIN
  will allow you to query the number of lines per invoice with:
\startXPLAIN
get invoice its lines.
\stopXPLAIN
  But there's no check that an invoice has at least one line. Such
  checks should be run at the end of a transaction, and very view
  databases allow such checks.
\item The level of support for the \Xplainkeyword{init} ({\it
  default}) statements depends on the particular dialect. Constant
  values are most widely supported. Literal values (including
  expressions that evaluate to a constant such as 1 + 1) need trigger
  support in certain dialects. All other forms require support for
  triggers.\par
  Due to some parsing peculiarities there should be no comment between
  a type and its \Xplainkeyword{init} (\Xplainkeyword{default})
  statements.\par
  {\it case} expressions are not supported at the moment.
\item The \Xplainkeyword{check} constraint is not supported.
\item Make sure the \Xplainkeyword{init} statement immediately follows
  the \Xplainkeyword{type} statement, without intervening other
  statements. This is due to the fact that {\tt xplain2sql} does not
  build an AST, but generates output immediately while parsing. This
  will be corrected in a future release.
\item Purging of attributes is supported but not all dialects support
  the {\bf alter table drop column} statement.
\item \Xplainkeyword{purge} of attributes or inits is not yet supported.
\item \Xplainkeyword{case} keyword in inheritance is not yet supported.
\item Because the date functions are not supported, any usage of a
  literal date is sql dialect specific.
\item Not all Xplain 5.8 string and mathematical functions are
  supported.
\item The \Xplainkeyword{newline} keyword is not supported.
\item The \Xplainkeyword{input} keyword is not supported.
\item Constants can be updated, although this is not allowed in Xplain.
\item The \Xplainkeyword{cascade} keyword is not supported.
\stopitemize


\subsection{General conversion to SQL issues}

There are six problems converting Xplain to SQL:
\startitemize[n]
\item If a dialect doesn't support double quoted identifiers, don't try to
  use an sql keyword as name for a type or base. Currently they're not
  translated or escaped.
\item Xplain data definition support can be converted quite well to
  SQL. Only {\it init} construct which refer to an attribute, need
  trigger or stored procedure support.\par
  For a really good implementation you need to be able to write a
  'before insert' trigger. Else a conversion is not possible or you
  need to do all your inserts through a stored procedure.\par
  InterBase should be able to support {\it init} quite well, but the
  implementation is severely lacking in this respect.\par
\item The {\it value} statement needs some temporary storage, local to
  the current transaction.\par
  \index{InterBase}InterBase and DB/2 don't have this, so they
  essentially only support single-user. The generated code need to be
  modified on the fly to support multiple users by creating tables
  with unique names or so.\par
  \index{Microsoft SQL Server}Microsoft SQL Server and
  \index{PostgreSQL}PostgreSQL support temporary storage.
\item The \Xplainkeyword{extend} statement also needs temporary
  storage, the problems and limitations are the same as for the
  \Xplainkeyword{value} statement.
\item Full support for the \Xplainfunction{some} function requires the
  ability to select a single row, usually the first, from a statement
  returning multiple rows. Not all dialects have this ability. If the
  \Xplainfunction{some} statement selects a single instance, i.e.:
\startXPLAIN
get some t "1" its a.
\stopXPLAIN
  the conversion is safe. Also if you know the result is just a single
  instance, the conversion should also be good enough.
\item It is not clearly defined if Xplain is
	\index{case-sensitive}case-sensitive, but it seems it is. The code
	that xplain2sql generates is case-sensitive on a database that is
	case-sensitive and \index{case-insensitive}case-insensitive on a
	database that is case-insensitive. This is considered a feature. For
	certain backends it is possible to generate case-sensitive code on a
	database that is case-insensitive, but a need for this option has
	not yet arisen. All relational database servers seem to be
	case-sensitive by default, except Microsoft SQL Server. With
	Microsoft SQL Server 2000, case-sensitivity can be set on a per
	database basis.
\stopitemize


\subsection{DB/2 6 limitations}

xplain2sql has the following limitations when converting to DB/2
version 6 (but \in{see also}[sec:db2_71]):
\startitemize[n]
\item Auto||generated primary keys are not supported in the version 6
  script. But it seems DB/2 version 6 had them too, so you might just
  want to use the 7.1 script against version 6 databases.
\item Not-literal {\it inits} are not supported.
\item There is no support for temporary tables. The SQL generated for
  the {\it value} statement is therefore not immediately useful in a
  multi-user environment.
\item For extends, one use either views (option \type{-extendview}) or
  subselects (option \type{-extendinline}) instead of extends with tables.
\item Insert, update and delete stored procedures cannot be generated
  as DB/2 only has Java or C stored procedures.
\item No full support for the {\it some} statement.
\stopitemize


\subsection[sec:db2_71]{DB/2 7.1 limitations}

xplain2sql's support for DB/2 is quite complete. These are the current
limitations when converting to DB/2:
\startitemize[n]
\item DB/2 7.1 cannot alter a temporary table, so the value statement
  cannot use one. The SQL generated for the \Xplainkeyword{value}
  statement is therefore not immediately useful in a multi-user
  environment.\par
  Inside procedures, values should work fine though.
\item Full support for the \Xplainfunction{some} statement is
  possible, but has not yet been implemented.
\item You need at least Fixpack 2 if you want to be able to specify
  instance identification values in insert. And note that the code
  generated by xplain2sql only works in the single-user case,
  i.e. when loading the data base with data.
\stopitemize

DB/2 output uses the '@' character as statement terminator. Use this
command to process xplain2sql generated scripts with db2:

\starttyping
db2 -td@ -vf myfile.sql
\stoptyping

Also the \index{db2upd}\type{db2upd} function must be present in your
\filename{sqllib/function} directory. That can be simply done with:

\starttyping
ln -s /usr/IBMdb2/V7.1/function/db2udp .
\stoptyping

Replace V7.1 with your DB/2 version of course.

You also need \index{DB/2 temporary table space}temporary table space
in your databases. I create this with:

\startSQL
CREATE USER TEMPORARY TABLESPACE USR_TEMPSPC1
  MANAGED BY SYSTEM USING ('usr_tempspc1')
\stopSQL

Certain SQL output needs access to a table which returns just one
row. xplain2sql uses SYSIBM.SYSDUMMY1. This is a default table that is
present on every installation and is to be used for exactly this
purpose. Users need to have select permissions on this table.


\subsection{FireBird 2.1 limitations}

xplain2sql has the following limitations when converting to FireBird 2.1:
\startitemize[n]
\item A large number (larger than 31 bits) which has a domain
  restriction will fail if the specified domain restriction exceeds
  31 bits. Example:
\startXPLAIN
base salary (I20) (5000..*).
\stopXPLAIN
\item Complex {\it inits} cannot be supported by FireBird. A complex
  init initializes an attribute using the value of another attribute
  through an its list like:
\startXPLAIN
init default entry its entry description = group its group description.
\stopXPLAIN
  It is not possible to generate SQL code for this with
  FireBird. Currently these \Xplainkeyword{init}s are not masked out for
  FireBird output and generate an exception. So don't use them in
  data definition scripts meant to be processed by FireBird.
\item No full support for the \Xplainfunction{some} function as it is
  not possible to specify that a select should return just a single row.
\item FireBird does not support the {\tt insert into} \dots\ {\tt
  default values} clause. If there is nothing to insert, xplain2sql
  will generate incorrect code. That can happen if there is an {\it
  init} on all attributes of a type.
\item An \Xplainkeyword{extend} inside a stored procedure does not
  work.
\item A procedure with no statements does not compile on FireBird.
\stopitemize


\subsection{InterBase 6/FireBird 1.x limitations}

xplain2sql has the following limitations when converting to InterBase
or FireBird 1.x:
\startitemize[n]
\item It's just too easy to create a query that bombs InterBase. This
  generator tries to output the most safe code, but if the output
  looks complex, it probably will not make InterBase happy.
\item A large number (larger than 31 bits) which has a domain
  restriction will fail if the specified domain restriction exceeds
  31 bits. Example:
\startXPLAIN
base salary (I20) (5000..*).
\stopXPLAIN
\item InterBase does not have support for temporary tables. The SQL
  generated for the {\it value} statement is therefore not immediately
  useful in a multi-user environment. Inside a procedure {\it value}
  should work perfectly.
\item Complex {\it inits} cannot be supported by InterBase. A complex
  init initializes an attribute using the value of another attribute
  through an its list like:
\startXPLAIN
init default entry its entry description = group its group description.
\stopXPLAIN
  It is not possible to generate SQL code for this with
  InterBase. Currently these \Xplainkeyword{init}s are not masked out for
  InterBase output and generate an exception. So don't use them in
  data definition scripts meant to be processed by InterBase.
\item No full support for the \Xplainfunction{some} function.
\item InterBase does not support the {\tt insert into} \dots\ {\tt
  default values} clause. If there is nothing to insert, xplain2sql
  will generate incorrect code. That can happen if there is an {\it
  init} on all attributes of a type.
\item InterBase does not have support for temporary tables. The SQL
  generated for the \Xplainkeyword{extend} statement is therefore not
  immediately useful in a multi-user environment because everyone will
  write to the same table.
\item An \Xplainkeyword{extend} with a function might be a bit slower
  than in other SQL dialects, because InterBase does not have a
  coalesce function. Therefore we need to do a second pass to update
  those cases where the calculated value is null and does not have the
  default value for that function.
\item An \Xplainkeyword{extend} inside a stored procedure does not
  work.
\item A procedure with no statements does not compile on InterBase.
\stopitemize


\subsection{Microsoft SQL Server limitations}

The Microsoft SQL Server implementation is very complete. On one hand
this is due to me using it a lot, on the other, because it is
absolutely the most easiest SQL dialect to work with. Many other
dialects require special syntax inside stored procedures, special
hacks, or just don't have the required functionality. Kudos to
Microsoft for this one!

The only conversion artefact is that a non literal \Xplainkeyword{init
default} is translated to a column which can be null. This is due to
the fact that in Transact SQL triggers only fire after the constraints
have been satisfied (incorrectly I think). There will be an on update
trigger in a future release to check that a user does not set such
attributes to null.


\subsection{MySQL 3 and 4 limitations}

MySQL is a very strange `database' among the other databases xplain2sql
supports. It doesn't obey the ACID properties, nor do its authors
think this is useful. Anyway, only the data definition part of Xplain
can be converted to MySQL. Other statements like the {\it extend}
statement, ask too much of what MySQL has to offer. As MySQL doesn't
support subselects, Xplain functions can also not be translated.

For MySQL 4 you must make sure ANSI mode has been enabled.

You might find the \from[mysqlgotchas] page useful.


\subsection{MySQL 5 limitations}

MySQL 5 has begun to look like a real database. That means the support
of xplain2sql is far more complete than for earlier versions. There
still remain a few issues though.

\startitemize[n]
\item The Xplain system default functions systemdate and loginname are
  not supported. Output is incorrect in this case.
\item Self referencing queries might not work due to problems with
  temporary tables, see \from[mysql-temporary-table-problems].
\item Specifying a non literal \Xplainkeyword{init default} will cause
  a column to allow null values. This is due to the fact that in
  MySQL triggers only fire after the constraints have been
  satisfied (incorrectly I think). There will be an on update trigger
  in a future release to check that a user does not set such
  attributes to null.
\stopitemize

xplain2sql has been tested with MySQL 5.0.15. Due to certain changes
with earlier 5.0 versions this is probably also the minimum 5.0
release xplain2sql supports.


\subsection{Oracle limitations}

\startitemize[n]
\item Domains are not translated to subtypes. Support for this would
  require to output a package with all types at the beginning of the
  output. This requires a change in how {\tt xplain2sql} emits its
  output. It probably will be possible in version 3.0.
\item \Xplainkeyword{extend} command not really supported. Works
  slightly outside stored procedures.
\item Output for Booleans should include restriction that it can be
  only 'T' or 'F'.
\item There is no support for more complex inits (those referring to
  other attributes).
\item Updating a column using the value of an extend will fail if
  there are no rows in the table.
\item Don't mix inserts that provide an instance id with inserts that
  rely on a auto-generated key. The current Oracle output does not update
  the generator when an instance id is provided.
\item Procedures (an {\tt xplain2sql} extension) are transformed to a
  function that returns a reference cursor (weak cursor). I'm not sure
  if that is correct. The goal is to produce output that can generate
  a result set that is recognized by ADO/ODBC drivers.
\item A procedure with no statements will not compile under Oracle.
\stopitemize

This is an example of how to call an Xplain generated function that
returns a result set:

\startSQL
variable c refcursor
exec :c := "sp_test"
print c
\stopSQL


\subsection[pgsql8]{PostgreSQL 8 limitations}

PostgreSQL 8 support is very complete. Remaining issues:
\startitemize[n]
\item Stored procedures cannot return multiple record sets. I.e. two
  gets in a stored procedure will lead to invalid code.
\item Names with spaces still have major issues. For example
  PostgreSQL auto generates a sequences for serial primary keys,
  The sequence name is derived from the column name. But if the column
  name contains spaces, it is not possible to retrieve the value of
  the generated key.\par
  The use \type{-nospace} to suppress spaces is therefore recommended.
\item Should emit \type{create type} statements for domains.
\item Constants are parsed as float8 and do not map to numeric
  datatypes.
\item If you use an identifier that is actuall an SQL keyword,
  PostgreSQL will not accept this SQL. PostgreSQL quoting doesn't
  actually quote keywords.
\stopitemize

Note that xplain2sql needs the `plpgsql' language loaded in the
database. This can be achieved by the {\bf create language} statement
or by using the {\tt createlang} utility. An example of loading this
language into database `mydatabase' is:

\starttyping
createlang --pglib=/usr/lib/pgsql plpgsql mydatabase
\stoptyping

Another option is to add this to your template database, so it will
always be present.

If a stored procedures returns a resultset (i.e. you have a {\it get}
statement in it), that result set can be accessed with:

\startSQL
select * from "sp_my procedure"();
\stopSQL

This result set works fine with ODBC. The output is different from
xplain2sql 2.0, which used an older technique. The new technique is
far more practical.

If you use the \type{-pgsql71} switch, code for the 7.1 dialect will
be emitted. For set returning functions this is slightly different
from the 7.3 code. Cursors can be accessed with:

\startSQL
begin;
select "sp_my procedure"();
fetch all in "<unnamed cursor 1>";
commit;
\stopSQL
%"

For ODBC applications I could get a result set if I first executed
this statement:

\startSQL
begin;
select "sp_my procedure"();
\stopSQL

And next executing, using the same connection and statement handle, this:

\startSQL
fetch all in "<unnamed cursor 1>";
\stopSQL
%"

No commit statement until you have read all rows. Execute commit

\startSQL
commit;
\stopSQL

before calling another set returning function. It's a bit unfortunate
that the next unnamed cursor will be called \type{"<unnamed cursor
2>"}. So that makes a complex application extremely awkward. If
someone knows how to generate named cursors, please.
%"

If you want to create a procedure in xplain2sql that is to be used as
a \index{PostgreSQL+trigger}, use {\it trigger procedure} instead of
just {\it procedure}. xplain2sql will emit the necessary code so the
procedure can be used as a trigger.


\subsection{PostgreSQL 7 limitations}

Since PostgreSQL 7.0 subselects are accepted. With that, it moves to
the forefront of supported SQL dialects. Remaining issues:
\startitemize[n]
\item Stored procedures cannot return multiple record sets. I.e. two
  gets in a stored procedure will lead to invalid code.
\item Names with spaces don't really work in functions in Postgres
  7.1. It seems quoted names are not handled properly by the plpgsql
  parser. Version 7.3 works fine here.
\stopitemize

See \in{section}[pgsql8] for a discussion on calling PostgreSQL SQL
code from a client.


\subsection{SQLite limitations}

SQLite was one of the easiest versions to support. Support is quite
complete, but be aware that SQLite does not actually guarantee all
specified constraints.

Unsupported are:
\startitemize[n]
\item No support for the extended Xplain command
  \Xplainkeyword{procedure}, because SQLite does not support stored
  procedures.
\item \Xplainkeyword{purge} of attributes is unsuported, because
  SQLite does not support this.
\item Specifying a non literal \Xplainkeyword{init default} will cause
  a column to allow null values. This is due to the fact that in
  SQLite triggers only fire after the constraints have been
  satisfied (incorrectly I think). There will be an on update trigger
  in a future release to check that a user does not set such
  attributes to null.
\stopitemize




\chapter{Extensions to Xplain}

This chapter describes the various enhancements xplain2sql has
compared to the original Xplain specification.

\section{More representations}

Beside the standard Xplain representations, see
table~\in[table:standard_representations], xplain2sql also has support
for additional representations, see
table~\in[table:additional_representations]. Support for additional
representations depends on the particular sql dialect.

\placetable[][table:standard_representations]{Standard Xplain representations}
\starttable[|l|p(11cm)|]
\HL
\VL Representation \VL Explanation \VL\SR
\HL
\VL Ax \VL Character field of length x. Translated to a VARCHAR.\VL\SR
\VL B \VL Boolean. Either translated to a Boolean if the dialect
    supports it, or to a CHAR(1) with 'T' and 'F' as allowed values.\VL\SR
\VL Ix \VL Integer field of length x. \VL\SR
\VL Rx,y \VL Float field with x positions before the decimal point and
y after. \VL\SR
\HL
\stoptable

\placetable[][table:additional_representations]{Additional
representations supported by xplain2sql}
\starttable[|l|p(11cm)|]
\HL
\VL Representation \VL Explanation \VL\SR
\HL
\VL C \VL Fixed length character field.\VL\SR
\VL D \VL Implemented as date and time field if dialect supports it,
  else it's the standard date field. \VL\SR
\VL M \VL Money field. \VL\SR
\VL P \VL Picture field. \VL\SR
\VL T \VL Text/memo field (unlimited compared to character field). \VL\SR
\HL
\stoptable


\section{Null/not null support}

By default attributes (columns) are not null. This is basically an
Xplain requirement as Xplain doesn't know Nulls. There is one
exception: memo and picture columns are Null by default. This has to
do with storage. Even the empty string takes up a full page in case of
a memo column.

To change the default Null behavior xplain2sql supports two additional
keywords : {\it optional} and {\it required}.

If you want a certain column to be null you specify the {\it optional}
keyword in front of it. So if business town is not required, you
write:

\startXPLAIN
type department	(A3) = department name, optional business_town.
\stopXPLAIN

On the other hand, if you want a certain column to be required, use
the {\it required} keyword. So if a employee should have a picture:

\startXPLAIN
type employee (A3) = name, home_town, salary, required photo.
\stopXPLAIN

Be aware that there is no real support for optional attributes, base
or type. If you say:

\startXPLAIN
get employee its department its department name.
\stopXPLAIN

employees without a department will be silently skipped.


\section{Quoted names}

It is sometimes useful to use an Xplain keyword in a name. This can be
done by quoting the name inside the ` and the ' characters like:

\startXPLAIN
unique index t2 its `first index' = a1.
\stopXPLAIN

In this example {\it index} is a reserved name. Because it is quoted,
it is no longer rejected.

Quoted names are allowed wherever a name is allowed.


\section{Unique support}
If you want a certain column to have a unique value for a table,
specify the {\it unique} keyword in front of it. So if a department
name is unique, you write:

\startXPLAIN
type department	(A3) = unique department name, business_town.
\stopXPLAIN


\section{Rename column heading support}

When a {\it get} statement is translated to the SQL {\bf select}
statement, the column names (column headings) do no longer indicate
how there were derived. So in SQL columns can have duplicate names,
which is not possible in Xplain. In such a case you can use the
{\it as} keyword to rename a column. Example:

\startXPLAIN
get department its department name as name.
\stopXPLAIN


\section[sec:index]{Index support}

It is possible to specify the indexes that should be created with the
keyword {\it index}. You can either create a normal {\it index}, a
{\it unique index}, a {\it clustered index} or a {\it unique clustered
index}.

Example:

\startXPLAIN
index department its firstindex =
  department name.

unique index department its firstindex =
  department name.

unique clustered index department its `some name' =
  business_town, department name.
\stopXPLAIN


\section{Including and using other script files}

For more complex environments one often wants to split Xplain
statements across several files. xplain2sql has two features to
support this quite well:

\startitemize[n]
\item xplain2sql can {\em include} another Xplain script. Example:
\startXPLAIN
database test.

.include "myinclude.ddl"

type something (I4) = name.
\stopXPLAIN
  A {\it .include} statement reads the contents of the included file as
  if it was typed right there. The definition of name for example is
  assumed to be in \filename{myinclude.ddl}. The definition of name and
  the definition of something are converted to SQL, and therefore
  both present in the output.
\item xplain2sql can {\em use} another Xplain script. Example:
\startXPLAIN
database test.

.use "myinclude.ddl"

type something (I4) = name.
\stopXPLAIN
  A {\it .use} statement reads the definitions of the used file, but
  does not generate any output for this file. In the above example the
  definition of something is converted to SQL, but the definition of
  name is not. So only the definition of something is present in the
  output.
\stopitemize

Other remarks:
\startitemize
\item Note that these dot commands start with a dot, but do not end with a
  dot as all Xplain commands do!
\item Both included and used files have one restriction: they should not
  have the database statement.
\item Files can be included and used in any combination, and without depth
 restriction. xplain2sql currently does not detect or warn for circular
 references!
\stopitemize


\section[sec:literalsql]{Literal SQL}

It is possible to include literal (include) SQL in an Xplain
file. That Xplain file however, will not be portable to a different
dialect anymore.

Literal SQL starts with the '{' character and stops with the '}'
character. Example:

\startXPLAIN
type department	(A3)  = department name, business_town.

{
create unique index idx_department on
  department("department name");
}
\stopXPLAIN

First a create table statement will be generated and the create index
statement will be directly behind it. Note that since release 0.8.1
you can specify indexes directly with enhanced Xplain, so a more
portable form would be:

\startXPLAIN
type department	(A3)  = department name, business_town.

unique index department its `idx_department' = department name.
\stopXPLAIN

See also \in{section}[sec:index].


\section{Stored procedures support}

The \Xplainkeyword{procedure} statement in xplain2sql makes it
possible to create SQL dialect independent stored procedures. A
procedure is a collection of one or more Xplain data retrieval
statements. Data definition statements are not supported and usually
not allowed by SQL dialects either.

The following example wraps a get statement inside the stored
procedure `retrieve names':

\startXPLAIN
base short text (A60).
base long text (T).

type name (I9) =
  unique short text, long text.

procedure retrieve names =

get name.

end.
\stopXPLAIN

Procedures can take parameters as well. Parameters can be base or type
names, with an optional role. An example of a base parameter is:

\startXPLAIN
procedure retrieve names by text with short text =

get name
  where
    short text = ?short text.

end.
\stopXPLAIN

A parameter can be used by using a question mark followed by its
name. The following example shows how a type parameter can be used to
select a single instance:

\startXPLAIN
procedure retrieve name instance with name =

get name ?name.

end.
\stopXPLAIN

It is possible to use literal SQL (see \in{section }[sec:literalsql])
inside a procedure. The following example demonstrates this:

\startXPLAIN
procedure `delete unreferenced content' =

extend content with `reference count' =
  count node
  per content.

update content "1" its `reference count' = 1.

# delete content
#   where `reference count' = 0.

# this code is extremely faster on TSQL
{
delete from [content]
  from [content]
  join [#content.reference count] on
    [#content.reference count].[id_content] = [content].[id_content]
  where
    [#content.reference count].[reference count] = 0
}

end.
\stopXPLAIN

The reason why the literal SQL appears in the procedure shown above is
a fault in Xplain. Xplain translates the commented out
\Xplainkeyword{delete} to a {\bf delete} with a subselect. But
subselects in Transact-SQL perform much worse than a select with
joins.

Currently, there is no support for more advanced constructs like {\it
if-then-else} or loops. There are also no provisions to call an Xplain
procedure from within Xplain. With literal SQL this can still be done,
but not portably.

xplain2sql supports two special stored procedures:
\startitemize[n]
\item If the procedure starts with ``recompiled'', the output on SQL
  Server includes a ``with recompile'' option so the plan is not
  cached. On other platforms the output is not changed.
\item If the stored procedure starts with ``trigger'', the output on
  PostgreSQL is a procedure that can be used as a trigger. The
  procedure has to conform to the PostgreSQL specification for
  triggers, so it has to return an explicit value, including null. On
  other platforms the output is not changed.
\stopitemize


\section{Enhanced auto-primary key support}

If you use auto-primary generation, you sometimes need access to the
last generated number. You can retrieve this number with {\it
inserted} followed by the type name. Example:

\startXPLAIN
insert customer *
  its name = '''Smith''.

value last id = inserted customer.

value last id.
\stopXPLAIN

The number is only valid, immediately after the insert. If the table
has triggers, the number may become invalid even in that case.



\chapter{Using xplain2sql in a legacy environment}

xplain2sql has various features that allow it to work with existing
databases, created by other tools than xplain2sql, or perhaps created
by hand. This chapter discusses the features xplain2sql has to offer
here.

In these environment you usually define the database again in Xplain,
but perhaps only use the procedure output of Xplain in the database.
So the existing database remains unmodified, but Xplain queries are
executed against it.


\section{Names with underscores}

Names in legacy databases frequently contain underscores, something
that Xplain sees as the end of a role. You can use quoting, an
extension to Xplain, that xplain2sql implements. Quoting an identifier
removes any special treatment and xplain2sql will just see it as a
single unit, not a role and name for example. An identifier is quoted
by prefixing it with an open single quote and suffixing it with a
close single quote.

Quoting example:

\startXPLAIN
type `legacy_name' (I9) = a, b.

get `legacy_name' its `a'.
\stopXPLAIN


\section{Primary keys}

xplain2sql automatically generates a primary key column consisting of
`id_' followed by the table name. xplain2sql puts the `id_' in front
because some output dialects have limits on the length of identifiers,
and with `id_' in front the `id_' gets preserved.

But in legacy systems you frequently find things like table name
followed by `_id'. You can use the \type{-pkformat} option to generate
the primary key columns in that case. Example using the Unix shell:

\startXPLAIN
xplain2sql -pgsql -pkformat '$$s_id' test.ddl
\stopXPLAIN

This will change the default format of xplain2sql to use the
\type{\$s_id} format. Note that because of shell escaping a double
\type{\$\$s} is used, in other environments \type{\$s} might suffice.


\section{Sequences}

When the format of the primary key is different, the sequence names
(used in PostgreSQL for example) are different as well. Use the
\type{-sequenceformat} option to change the format. Example:

\startXPLAIN
xplain2sql -pgsql -sequenceformat '$$s_seq' test.ddl
\stopXPLAIN

\type{-sequenceformat} takes one or two parameters. If only one
\type{\$\$s} is present this is assumed to be the primary key name of
the table. If two are present, they are replaced by the table namd and
primary key name respectively.


\section{Use and include}

Use \type{.use} to include files without generating SQL code.


\section{Creating views to map xplain2sql output more closely}

You can create views in SQL to map tables to something that more
closely resembles the output of xplain2sql. \dots


\section[sec:external-functions]{Calling functions}

xplain2sql allows you to call arbitrary functions using the \$
syntax. Example:

\startXPLAIN
get subcomponent its
    component,
    component its description,
    $`getcomponent_weight' (component) as `total_weight',
    quantity,
    $`getcomponent_price' (component) as `cost'
  where
    master_component = ?component
  per
    component its description.
\stopXPLAIN

This calls two functions, \type{getcomponent_weight} and
\type{getcomponent_price}. Parameters are normal Xplain expressions
and should be valid. But they are passed as is, it is up to you to
make sure the number of arguments and the argument type is
correct.


\section{Injection of arbitrary SQL code in expressions}

It is possible to inject arbitrary SQL between statements as
explained in \in{section}[sec:literalsql]. But xplain2sql also supports
injection of SQL everywhere an Xplain expression is expected, for
those times you really need to modify the output.

The following example shows a procedure that returns a certain quote.
It returns a valid until date using a call to a PostgreSQL function to
format it properly. However, the valid until date is the date of the
quote plus a specific number of days, passed in as the interval
parameter.

\startXPLAIN
procedure document quote with quote, interval =

get quote ?quote its
    $`to_char'(`date_quote' + {cast(a_interval as interval)}, "Month DD, YYYY") as `valid_until'.

end.
\stopXPLAIN
%$

Xplain nor xplain2sql has any functions to format dates (nor should
it), but for legacy support it is sometimes easiest to work with the
flow and existing code. Natively Xplain doesn't even have a date
format, although xplain2sql has a not well-specified date type.

But xplain2sql doesn't have any date functions, and it would be
impossible to support them among all the dialects. PostgreSQL has lots
of date functions, so this piece of code calls the PostgreSQL cast
function to cast a_interval, an integer, into a number of days using
PostgreSQL's interval data type. Calling cast as a function as
explained in \in{section}[sec:external-functions] wouldn't work
because something as \type{\$cast(a_interval as interval)} is not
valid Xplain. That's where the SQL injection comes into play.

With SQL injection you are really on your own. You have to make sure
it works with the code Xplain generates.




\chapter{XML output}

When using xplain2sql to generate output for various dialects, the
names of tables and attributes in those dialects can differ. For
example, certain dialects require quoted identifiers when they have
been declared with quotes. Other dialects don't support quoting at
all, or have non-ANSI quoting conventions. This makes it hard to write
portable code. With xplain2sql's XML output, it becomes possible to
access the generated table and column names without introducing subtle
inconsistencies in the source.

\section{xplain2sql.xml file}

xplain2sql will write an XML description of the generated SQL when
supplied with the {\tt -xml} option. The XML file is called
\filename{xplain2sql.xml}. This XML file contains the bases, types and
procedures that are generated. This XML file can be used by XSLT
scripts, or other tools, to generate middleware code, or just to
extract the proper names to use for creating on-the-fly SQL.

Take for example the following Xplain:

\startXPLAIN
base customer name (A40).
base address (A40).

type customer (I4) = customer name, address.
\stopXPLAIN

When xplain2sql is called with the -xml flag:

\starttyping
xplain2sql -xml sample.ddl
\stoptyping

The generated XML looks like this (it is slightly edited for readability):

\startXML
<?xml version="1.0" encoding="ISO-8859-1" ?>
<sql>
  <table xplainName="customer" xplainDomain="(I4)"
        sqlName='"customer"'
        identifier="customer"
        sqlNameAsEiffelString='%"customer%"'
        sqlNameAsCString='\"customer\"">
    <column xplainName="customer" xplainDomain="(I4)"
        sqlName='"id_customer"' sqlType="smallint"
        identifier="customer"
        sqlNameAsEiffelString='%"id_customer%"'
        sqlNameAsCString='\"id_customer\"'
        init="none"/>
    <column xplainName="customer name" xplainDomain="(V40)"
        sqlName='"customer name"' sqlType="character varying(40)"
        identifier="customer_name"
        sqlNameAsEiffelString='%"customer name%"'
        sqlNameAsCString='\"customer name\"'
        init="none"/>
    <column xplainName="address" xplainDomain="(V40)"
        sqlName='"address"' sqlType="character varying(40)"
        identifier="address"
        sqlNameAsEiffelString='%"address%"'
        sqlNameAsCString='\"address\"'
        init="none"/>
  </table>
</sql>
\stopXML

For every type a \XMLtag{table} tag is generated. In this tag there
are one or more \XMLtag{column} tags.

Every XML tag contains three things:
\startitemize[n]
\item The Xplain name and type.
\item The SQL name and type (if applicable).\par
    If the SQL name is quoted, the name includes those quotes as many
    SQL dialects regard quotes as part of the name.
\item Names useful in programming languages which try to access the
  database.\par
  The `identifier' attribute can be used as the name of a variable for
  example. Any spaces occurring in `xplainName' are replaced by
  underscores.\par
  `sqlNameAsEiffelString' and `sqlNameAsCString' contain the name as
  known in the SQL dialect. They contain the proper quoting for Eiffel
  and C respectively. Usually XSLT scripts are used to transform this
  XML and it's a real pain to detect and do proper quoting there, so
  that's why xplain2sql already does these things.
\stopitemize

Another example might clarify the quoting. Let's translate the Xplain
example to Transact-SQL.

\starttyping
xplain2sql -xml -tsql sample.ddl
\stoptyping

The file \filename{xplain2sql.xml} now looks like this:

\startXML
<?xml version="1.0" encoding="ISO-8859-1" ?>
<sql>
  <table xplainName="customer" xplainDomain="(I4)"
        sqlName="[customer]"
        identifier="customer"
        sqlNameAsEiffelString="[customer]"
        sqlNameAsCString="[customer]"
        spDelete="[sp_Deletecustomer]"
        spDeleteAsEiffelString="[sp_Deletecustomer]"
        spDeleteAsCString="[sp_Deletecustomer]"
        spInsert="[sp_Insertcustomer]"
        spInsertAsEiffelString="[sp_Insertcustomer]"
        spInsertAsCString="[sp_Insertcustomer]"
        spUpdate="[sp_Updatecustomer]"
        spUpdateAsEiffelString="[sp_Updatecustomer]"
        spUpdateAsCString="[sp_Updatecustomer]">
    <column xplainName="customer" xplainDomain="(I4)"
        sqlName="[id_customer]" sqlType="smallint"
        identifier="customer"
        sqlNameAsEiffelString="[id_customer]"
        sqlNameAsCString="[id_customer]"
        init="none"/>
    <column xplainName="customer name" xplainDomain="(V40)"
        sqlName="[customer name]"
        sqlType="[Tcustomer_name]"
        identifier="customer_name"
        sqlNameAsEiffelString="[customer name]"
        sqlNameAsCString="[customer name]"
        init="none"/>
    <column xplainName="address" xplainDomain="(V40)"
        sqlName="[address]" sqlType="[Taddress]"
        identifier="address"
        sqlNameAsEiffelString="[address]"
        sqlNameAsCString="[address]"
        init="none"/>
  </table>
</sql>
\stopXML

\section{Processing the XML file with XSLT}

xplain2sql comes with some XSLT scripts that turn the generated XML
file into Delphi or Eiffel classes. For example the
\filename{ecli_stored_procedure.xsl} generates an Eiffel class that
allows one to call a stored procedure and get access to the parameters
of that procedure and the resulting output.

Usage:

\starttyping
Xalan \
  -o mw_retrieve_currencies.e \
  -p procedureName "'retrieve currencies'" \
  xplain2sql.xml ecli_stored_procedure.xsl
\stoptyping

The name of the procedure for which a class is generated is passed as
parameter. Note that the stylesheet
\filename{ecli_stored_procedure.xsl} is an example. It is fully
functional, but nothing should stop you from adapting and refining it
to suit your needs. That's why xplain2sql generates XML in order to
make such customisations possible.

xplain2sql also comes with an example Makefile script,
\filename{make_ecli_stored_procedures.xsl}, that generates a makefile
that will call \filename{ecli_stored_procedure.xsl} for every stored
procedure.




\chapter{Implementation}

Xplain is written in Eiffel. This chapter explains how xplain2sql was
designed and implemented.


In figure~\in[chart:flow] the flow of data through xplain2sql is
depicted. In goes a file with Xplain commands, out comes SQL code.

\placefigure
	[]
	[chart:flow]
	{High-level overview of xplain2sql workings}
	{\FLOWchart[flow]}

The actual behavior is more cyclic. The input file is not first all
tokenized, and next parsed. No, the parser tokenizes as much as is
needed for a certain construct, and next generates SQL code for that
construct. Than it starts tokenizing, parsing and outputting again.

xplain2sql is a prime example and implementation of the Builder
pattern (see~\in[Gamma95]). The {\bf Director} participant is
implemented by the tokenizer and parser in \EClass{XPLAIN_SCANNER} and
\EClass{XPLAIN_PARSER}. The {\bf Builder} participant is the class
\EClass{SQL_GENERATOR}.

In the following sections the tokenizing, parsing and sql generation
are discussed in detail, by showing how the Builder pattern is used.


\section{Tokenizing}

The tokenizer is the first stage of the Director participant. The
tokenizer class is written with Lex, see \filename{xplain_scanner.l}. This
file is converted by \filename{gelex} to the Eiffel class
\EClass{XPLAIN_SCANNER}. The responsibility of the tokenizer is to
output tokens.

However, the tokenizer knows a bit about sql generation, because of
inline sql and sql comments.

A `\{' starts literal sql. The tokenizer doesn't bother the parser
with this (literal sql can occur anywhere currently) and writes
the sql code straight to the output file. Literal sql of course makes
an Xplain file not portable. Maybe there should be support for sql dialect
specific literal sql with something like `\type{interbase}\arg{...}'.

A `\type{--}' starts a one line comment that is also written to the
output file. As not all sql dialects support one line comments, the
tokenizer calls the sqlgenerator (see section~\in[sec:sqlgeneration])
directly so it can transform a one line comment to a multi line like
comment. Multi line comments seem to be supported by all sql dialects,
so the tokenizer writes these straight to the output file.

There's one problem that needs to be solved someday. The default case
is matched, but it shouldn't.


\section{Parsing}

The parser is the second stage of the Director participant. The parser
is implemented in type|xplain_parser.y|. This is Yacc (with the Bison
extensions \type{%type}), which is converted by \type{geyacc} to an
Eiffel class. The responsibility of the parser is to parse Xplain
input and to give only correct Xplain to the sql generation routines.

The parser doesn't build a full Abstract Syntax Tree (AST), but only
partial ones. As soon as a construct has been parsed, the sqlgenerator
is called. In terms of the Builder pattern, the {\bf Products} that
the Builder creates  are XPLAIN_... classes.

Within the Directory, the Interpreter pattern
(see \cite[citeulike:115158]) is used a lot. Simple examples are the
parsing of representations, or domain restriction.

An advanced application of the Interpreter pattern is the expression
parsing part, see the partial expression inheritance hierarchy in
figure~\in[object:xplain_expression]. The {\bf Abstract\-Expression}
participant is implemented by \EClass{XPLAIN_EXPRESSION}. An example
of {\bf Terminal\-Expression} is \EClass{XPLAIN_STRING_EXPRESSION}. An
example of {\bf Non\-terminal\-Expression} is
\EClass{XPLAIN_\-INFIX_\-EXPRESSION}.

\placefigure
	[top]
	[object:xplain_expression]
	{Partial Xplain expression inheritance hierarchy}
	{\externalfigure[xplain_expression.png][width=\textwidth]}

The {\bf Context} participant is \EClass{XPLAIN_UNIVERSE}. This class is
implemented as a Singleton (see \cite[citeulike:115158]). This class
holds all objects which have a global scope (see
figure~\in[object:xplain_object]. Only \EClass{XPLAIN_EXTENSION} is
not stored in \EClass{XPLAIN_UNIVERSE} because the extension scope is
only known within a type.

\placefigure
	[]
	[object:xplain_object]
	{Xplain objects inheritance hierarchy}
	{\externalfigure[xplain_object.png][width=\textwidth]}

There are two {\bf Client} participants: the Client
\EClass{XPLAIN_PARSER} builds the AST, while the Client
\EClass{SQL_GENERATOR} (see figure~\in[object:sql_generator]) is given
the AST.

The AbstractExpression \EClass{XPLAIN_EXPRESSION} has several
Interpret methods. Method sqlvalue returns the sql code for that
particular expression as converted to be used in a sql select
statement. sqlinitvalue does the same, but is specifically for init
statements. See section~\in[sec:sqlgeneration] for how the sql code is
obtained.
AS usual, the parser also embellishes the partial AST with type
information. For this conversion, this could be kept quite simple. An
example of this is found when parsing a name (or names separated by
{\it its}) in an expression. Example:

\startXPLAIN
get employee its department its business_town.
\stopXPLAIN

The `department {\it its} business_town' is an example of an
\EClass{XPLAIN\-_ATTRIBUTE\-_EX\-PRE\-SSION}. But when parsing, we
don't know this beforehand of course. So the parser initially accepts
anything a user has typed. So it would initially accept `its not_an
its attribute'. The name or names are stored in a linked list of
\EClass{XPLAIN_ATTRIBUTE_NAME}s. There after, every name in this
linked list is supplied with with its type in
get_object_if_valid_tree, a routine in \EClass{XPLAIN_PARSER}. This
routine returns the object for the first name in this linked list of
node. We ask this object to create the expression for us. If the
object is a variable, it returns an
\EClass{XPLAIN_VARIABLE_EXPRESSION}. If it's a value, it returns an
\EClass{XPLAIN_VALUE_EXPRESSION}. If it's an attribute, it returns an
\EClass{XPLAIN_ATTRIBU\-TE_EXPRESSION}.

The parser (and all other parts of xplain2sql) use only one type of
data structure: a single linked list. This could be optimized a bit,
for example in \EClass{XPLAIN_UNIVERSE} to make lookups a bit faster,
but works well. The generic linked list is defined in
\EClass{XPLAIN_NODE}. For each kind of linked list a separate class is
defined. Examples are \EClass{JOIN_NODE},
\EClass{XPLAIN_\-ATTRIBUTE_\-NODE} or
\EClass{XPLAIN_\-ATTRIBUTE_\-NAME_NODE}.


\section[sec:sqlgeneration]{SQL generator}

The SQL generator is the Builder participant of the Builder pattern
and is implemented in the class \EClass{SQL_GENERATOR}.  The actual
sql dialect generation is done by the {\bf Concrete\-Builder}
participants, namely the descendants of \EClass{SQL_GENERATOR}, see
figure~\in[object:sql_generator]. \EClass{SQL_GENERATOR} is a deferred
class, doing the lowest common denominator stuff, descendants fill in
the details.

\placefigure
	[]
	[object:sql_generator]
	{SQL Generator inheritance hierarchy}
	{\externalfigure[sql_generator.png][,width=\textwidth]}

As suggested by BM in \cite[OOSC2] (chapter 24.3) the handle
technique is used to implement the \EClass{SQL_GENERATOR} class. A
different approach could have been to add all sql generating code to
\EClass{XPLAIN_PARSER} and inherit from \EClass{XPLAIN_PARSER} to get
the various dialects. But the handle approach is preferred. So
\EClass{XPLAIN_PARSER} doesn't know a single bit about generation sql
code, it just calls \EClass{SQL_GENERATOR} through its sqlgenerator
feature at appropriate times. This approach looks like the Bridge
pattern (see \cite[citeulike:115158]), but it would be a real Bridge if
the interface between \EClass{XPLAIN_PARSER} and
\EClass{SQL_\-GE\-NE\-RA\-TOR} was abstracted, which isn't the case.

The code generation within \EClass{SQL_GENERATOR} itself is divided in
two parts: the write_ methods are the only methods called by
\EClass{XPLAIN_PARSER}. The write_XXXX methods check if certain options
are enabled or supported. If so, they call the corresponding
create_XXXX which outputs the sql code.

The relation between the Product (for example \EClass{XPLAIN_TYPE})
and the Builder is complex if you look at how and when they call each
other, but there responsibilities are clear. the Product does know
nothing about sql generator, it only knows about Xplain
representations. The Builder itself knows a lot about Product, but to
generate certain code, it calls Product. Product next calls the
Builder back at some other method to get the correct sql.

A good example is the creation of domains or user defined data
types. In Xplain you define a domain with:

\startXPLAIN
base name (A30).
\stopXPLAIN

The corresponding sql code is something like:

\startSQL
create domain name as character(30).
\stopSQL

The Builder accepts the Product base (\EClass{XPLAIN_BASE}). To output
the representation of a domain, it asks the representation property of
\EClass{XPLAIN_BASE} to give it the correct sql code. Depending on the
representation, they call \EClass{SQL_GENERATOR} (or its descendants)
back in methods starting with the name sqldatatype_XXXX. In this
particular case sql\-data\-type_char is called. This happens
continually. Creating check constraints is done by asking the Product
domain restriction (\EClass{XPLAIN_DOMAIN_RESTRICTION}) for its
sqlcolumnconstraint. The actual domain restriction will call
\EClass{SQL_GENERATOR} back in its sqlcheck_XXXX methods.


\section{Other details}

In the Builder pattern a separate client is discussed which creates
the Directory (\EClass{XPLAIN_\-PARSER}) and configures it with the
desired Builder object (\EClass{SQL_GENERATOR}). There is no separate
client with xplain2sql. That part is handled by the execute method of
the \EClass{XPLAIN_\-PARSER} itself.




\chapter[sec:Xplain]{What is Xplain}

Xplain is a non-procedural data definition and manipulation language
invented by Johan ter Bekke and completed in 1983. xplain2sql is based on
the most recent definitions of the language, see for example
\cite[Bekke1992]. Xplain also refers to the Xplain product, a full
implementation of a semantic database, but in this document Xplain
always refers to the language.

This chapter gives a short introduction to Xplain, with the goal to
get people who don't know Xplain getting interested to learn
more. A good start is \from[bekkehome]. All material in this
section has been taken without getting permission :-)


\section{The Xplain book}

For English readers, the best introduction to Xplain is given in
\cite[Bekke1992]. I absolutely recommend this book, because it not
only introduces Xplain, but it gives fundamental advise about database
design and modeling. The semantic way of thinking has helped me a lot
to tackle large design problems and create excellent database
designs.

Abstract of this book (taken from Ter Bekke's website
\from[bekkehome]):

\startnarrower
The book explains the fundamental concepts and general principles of
data modeling, with practical cases to illustrate the theory where
appropriate. Recent developments in the database area have been
included. The book is organized in four main parts:

\startitemize
\item {\sl Overview of the discipline}, including an assessment of the
  relational theory. An overview of seven modern data modeling
  approaches is also presented in this part.
\item {\sl Fundamentals of data modeling}, introducing semantic concepts
  leading to proper object modeling.
\item {\sl Data modeling}, illustrated with numerous practical
  examples. Conversion into suitable traditional models (including
  relational), by applying just a few simple rules, makes the
  collection of data and query structures reliable and easier to
  understand.
\item {\sl Case studies}. Semantic data modeling is illustrated with three
  large cases. They illustrate data modeling in complex situations and
  the problem of formulating queries in practical environments.
\stopitemize

Relational systems have become widely accepted the last few
years. However, many pitfalls have also been discovered in the
relational theory.  This book presents an in-depth analysis of the
problems and offers a deeper understanding. By putting emphasis on the
semantic structure of a database, reliable solutions are created for
both data modeling and data manipulation problems. The theory is based
on both theoretical and practical research. It is illustrated with
many examples and exercises. Semantic Data Modeling offers a sound
basis for an education in modern data modeling techniques.
\stopnarrower


\section{Short introduction to Xplain}

In \cite[Bekke1996] a short introduction to Xplain is given,
before showing how Xplain can be used to model successive events,
version management in that particular case.


\subsection{Introduction}

The semantic abstractions aggregation and generalization appeared for
the first time in the database literature in a series of papers by
Smith and Smith (\cite[Smith77a], \cite[Smith77b],
\cite[Smith78]). These abstractions are considered to be
suitable for modeling complex situations in which different types of
relationships occur. Because of their nature, they are especially
useful for modeling hierarchical relationships. Many examples can be
found in literature. Considering these application areas, only minor
differences can be discovered between semantic data models and other
data models (e.g. relational and entity-relationship data
models). This is not a surprise:earlier hierarchical and network data
models already enabled us to find solutions for such situations.

The advantages of generalization appear especially in situations with
many irregularities, alternatives or exceptions. These new aspects
were not fully covered by other data models. It gave semantic models a
clear advantage over other existing data models.

In this paper the advantages of semantic abstractions are further
extended. Although semantic concepts enable us to create data models
for ordering and sequencing, they are hardly mentioned in
literature. This is also caused by the difficulties data models have
with modeling time aspects. It is also illustrated with the phrases
{\sl is-part-of} and {\sl is-a}, in other data models often general
need for this extended functionality appeared also in papers
discussing characteristics that must be satisfied by the next
generation of database systems (literature, see original article).

Sequencing is also important in the area of object oriented
databases. Examples can be found in literature (literature, see
original article). However, often extended entity-relationship
diagrams, flow diagrams or event-condition-action diagrams are used
for the purpose. These specifications are far from complete; they must
be accompanied with many informal procedural descriptions. These
specifications/descriptions are inadequate for development of the
required software.

Several commercial products support some form of version management
(for example: Objectivity/DB,Itasca, ObjectStore, Ontos and
Versant). However, a standard set of features for version management
is lacking (literature, see original article).

This paper presents extensions in the usage of semantic
abstractions. Both aggregation and generalization can be used for
modeling ordering and sequencing. Because time is involved, also the
phrases {\sl was-part-of} and {\sl was-a} should be used instead of
only the phrases is-part-of and is-a. The new opportunities will be
illustrated with several practical examples.

The resulting high-level specifications can be used for
standardization purposes. They can by implemented in any programming
environment (e.g. C or C++) or database environment (e.g. relational
or object oriented). For a better understanding of the opportunities,
first a short introduction to the underlying semantic concepts is
given.

\subsection{Abstractions}

This section contains a global overview of the concepts for semantic
data modeling using well-known examples. Each object will be visualized
explicitly by clearly distinguishing between identification and
descriptive properties. The resulting data models gain in semantic
contents as a consequence, while ambiguities and contradictions in the
specification are avoided. Only three fundamental abstraction types
with clear graphical equivalences in the structural diagrams are
required to guarantee inherent semantic integrity. They make use of
the fundamental {\sl type-attribute relationship}.

The real world is described by considering types of relevant objects, a
type being defined as a fundamental notion. The abstraction leading to
a type is called {\bf classification}. The examples (i.e. instances)
occurring in a database and required for the recognition of a type are
purely illustrations of the concept. The type is not being defined
hereby. Types are represented by rectangles in diagrams, see
figure~\in[fig:classification]. The counterpart of classification is
called {\bf instantiation}.

\placefigure
  []
  [fig:classification]
  {Classification}
  {\startMPcode input classification.mp \stopMPcode}

{\bf Aggregation} is defined as the collection of a certain number of types
in a unit, which in itself can be regarded as a new type (note the
analogy with the mathematical set concept). A type occurring in an
aggregation is called an attribute of the new type.

Aggregation allows view independence: we can discuss the obtained type
(possibly as a property) without referring to the underlying
attributes. By applying this principle repeatedly, a hierarchy of
types can be set up. An example is given in
figure~\in[fig:aggregation]. Normally the hierarchy contains only
aggregated types. Aggregation is indicated by a line connecting the
centers of two facing rectangle sides, while the aggregate type is
(according to its definition) placed above its attributes. Of course,
aggregation also has its counterpart: the description of a type as a
set of certain attributes is called {\bf decomposition}.

\placefigure
  []
  [fig:aggregation]
  {Aggregation hierarchy}
  {\startMPcode input aggregation.mp \stopMPcode}

A type is defined  by listing its attributes, so we could have the
following type definitions:

\startXPLAIN
type transport    = vehicle, destination, delivery_date, cargo.
type vehicle      = manufacturer, model, price, fuel,
                    construction_year.
type destination  = client_name, address, city, telephone number.
\stopXPLAIN


An example to illustrate the database contents is in \in{table}[db-contents]:

\placetable
  []
  [db-contents]
  {Database contents}
{\starttable[|l|l|l|l|l|l|]
\HL
\VL transport \VL
\VL vehicle \VL destination \VL delivery_date \VL cargo \VL\SR
\HL
\VL t1 \VL
\VL v1 \VL d2 \VL 19961206 \VL paper \VL\SR
\VL t2 \VL
\VL v3 \VL d4 \VL 19961207 \VL milk \VL\SR
\HL
\stoptable}


Type definitions carry semantics; they contain the essential
properties (e.g. uniqueness of the identifications t1 and t2 in the
table above) and essential relationships (the related vehicles v1 and
v3 and the related destinations d2 and d4 must occur in related
tables). Aggregation can be described using the verb to
have. According to the above type definition, a vehicle has a
manufacturer, model, price, fuel and
construction_year. Identifications are properties denoted by type
names (see table 1 above). This interpretation implies singular
identifications. Attributes (not types!) may contain roles. An example
is `construction_year' related to type `year'. Roles are separated
from the type by an underscore. Spaces are irrelevant in type
definitions.

The third kind of abstraction, important to conceptual models, is
called {\bf generalization}; it is defined here as the recognition of
similar attributes from various types and combining these in a new
type (note the analogy with the intersection operation from
mathematical set theory). We can equally discuss the new type without
mentioning the underlying attributes, and it can in itself again serve
as a property (i.e. it allows view independence).

Example: consider manufacturer, model, price, fuel, construction_year,
cabin, weight, wheels, power and coupling. The corresponding type is
truck. Consider, in addition, manufacturer, model ,price, fuel,
construction_year, chassis, seats and color, where the type might be
car.

The common attributes of the two types are: manufacturer, model,
price, fuel, construction_year. If required, these attributes result
in a new type "vehicle", which may be regarded as the generalization
of truck and car. Generalization can be represented in abstraction
hierarchies, as we have seen in the case of aggregation. This is shown
in figure~\in[fig:generalization].

\placefigure
  []
  [fig:generalization]
  {Generalization hierarchy}
  {\startMPcode input generalization.mp \stopMPcode}

In abstraction hierarchies, generalization is schematically
represented by a line connecting facing corners of rectangles, the
generalized type being placed below the specialized
ones. Generalization's counterpart (i.e. the union of attributes from
different types) is called {\bf specialization}.

In figure~\in[fig:generalization] we placed truck and car one behind
the other, while only one line connects to vehicle. This is the usual
representation of disjoint specializations --- i.e. a vehicle can be
either a truck or a car, but not both. The combination of a group of
disjoint specializations is called a block, so truck and car
constitute a block. Not all vehicles need to be specialized; an
example would be a motorcycle occurring only as instance of vehicle.

The generalization, together with the attributes to be added to it, is
(by definition of the concept) described in the type definition of the
specialization. So the type definitions are:

\startXPLAIN
type vehicle  = manufacturer, model, price, fuel, construction_year.
type truck    = [vehicle], cabin, weight, wheels, power, coupling.
type car      = [vehicle], chassis, seats, doors.
\stopXPLAIN

An example of the database contents is given in
\in{table}[db-contents]. This structure imposes uniqueness of
attributes related to generalizations. Besides that, values for these
attributes may occur only once in a block (i.e. v1 and v3 may not
occur as values in the corresponding truck table).

Generalization is commonly associated with the verb {\sl to
be}. According to the above type definitions, a truck is a vehicle
with cabin, weight, wheels, power and coupling, while a car {\sl is a}
vehicle with chassis, seats and color. The introduction of new
identifications for specializations (e.g. c1 and c2 above) makes
generalization hierarchies {\sl non-transitive dependent}.

The introductory definitions of aggregation and generalization above
have already clearly demonstrated the hierarchical character of these
abstractions. This will be elaborated in the following sections with
an emphasis on ordering aspects.


\section{Employee and department}

In chapter 6 of \cite[Bekke1992], a simple Xplain data model is
given, see figure~\in[fig:employee]. It defines the types department
and employee. Every employee belongs to a department, and a department
can have 0 or more employees.

\placefigure
  []
  [fig:employee]
  {Employee hierarchy}
  {\startMPcode input employee.mp \stopMPcode}

The database definition is:

\startXPLAIN
base department name (A30).
base town (A30).
base name (A30).
base salary (R9,2).

type department	(A3)  = department name, business_town.
type employee (A3)    = name, home_town, department, salary.
\stopXPLAIN

We can now ask certain questions (all taken
from \cite[Bekke1992]) and demonstrate a feature of Xplain:

\startitemize[n]
\item Select data of the employee with the identification E3.
\startXPLAIN
get employee "E3".
\stopXPLAIN

\item Select employees living in Guilding.
\startXPLAIN
get employee its name, department
    where home_town = "Guilding".
\stopXPLAIN

\item Select commuters.
\startXPLAIN
get employee its name, home_town, department
    where home_town <> department its business_town.
\stopXPLAIN

\item How many employees work in Guilding?
\startXPLAIN
get count employee
    where department its business_town = "Guilding".
\stopXPLAIN

\item What is the sum of all salaries?
\startXPLAIN
get total employee its salary.
\stopXPLAIN

\item What is the highest salary?
\startXPLAIN
get max employee its salary.
\stopXPLAIN

\item Are there any employees earning more than 50.000?
\startXPLAIN
get any employee
    where salary > 50000.
\stopXPLAIN

\item Select the name of an arbitrary employee in the Purchase department.
\startXPLAIN
get some employee its name
    where department its department name = "Purchase".
\stopXPLAIN

\item Provide an overview of the departments, including the number of
employees.
\startXPLAIN
extend department with numberofemployees=
    count employee
    per department.

get department its
    department name,
    business_town,
    numberofemployees.
\stopXPLAIN

\item Select departments with more than 100 employees.
\startXPLAIN
get department its
    department name,
    business_town
  where
    numberofemployees > 100.
\stopXPLAIN

\item Find the number of departments with more than 100 employees.
\startXPLAIN
get count department
  where numberofemployees > 100.
\stopXPLAIN

\item Which department has the most employees?
\startXPLAIN
value maximum =
    max department its numberofemployees.

get department its
    department name,
    business_town
  where
    numberofemployees = maximum.
\stopXPLAIN

\stopitemize

All these questions have two basic forms: if you go down in an Xplain
model (from employee to department), you use the \Xplainkeyword{get}
statement. But if you go up in an Xplain data model (from department
to employee), you first use the \Xplainkeyword{extend} statement and
next a \Xplainkeyword{get}.

And as you see, all queries are clear and easy to understand. The
corresponding SQL code is far harder to read.


\section{Case study: Bank}

In \cite[Bekke1992] three case studies are given. Parts of the
first case study are given here. Also there is an example file
supplied with the xplain2sql distribution, \filename{bank.ddl} which contains
all data definition and data manipulation presented in this case
study.

\subsection{Case description}

Consider a banking organization consisting of a few regional head
offices and numerous branches, each branch reporting to one of the
head offices. Customers can have various accounts with one of the
offices: saving accounts (with conditions), current accounts (with
cheque cards, Eurocheques, credit limits), business accounts (with
credit commissions, credit limits) and mortgage accounts (with
collection indications, redemption methods, redemption accounts). The
following account information is recorded: type, currency, sum,
transaction date and interest. Customers may deposit shares and bonds
with a branch for a fee. Deposits are linked to customers'
accounts. The recorded share information is name, number and nominal
value, while for bonds this name, nominal value, starting number and
end number. Customers can have multiple series of one bond in one
deposit.

\subsection{Assignment}

Provide a semantic model for above environment and indicate the
relevant attributes. Also provide the abstraction hierarchy. Indicate
some applications of the model by using a query language.

\subsection{Conceptual design}

Although the relationships between head offices and branches have been
described precisely, many designers find them difficult to
interpret. The model should obviously contain a type defining the
information about offices, because customers have accounts there. The
difference between head offices and branches is apparent from the fact
that branches belong to head offices; they are disjunct types in a
block. There are, however, additional relationships between the
offices:

\startXPLAIN
type office       =..., case
type head office	= [office], ...
type branch	      = [office], ..., head office.
\stopXPLAIN

Accounts play a paramount role in the remainder of the case study. The
specializations of the type account are all obvious. They have common
and separate attributes and the individual types are disjunct. The
attribute redemption account is the only potential difficulty in the
definition of the mortgage account. The situation is comparable to the
relationship between head offices and branches. If an individual
mortgage account is also related to another instance of the type
account, the second account could again be of the type mortgage
account, which is undesirable.

The relationship between the mortgage and redemption accounts is
therefore at the specialization level. Considering the four
alternatives (mortgage, business, current and saving accounts), the
only relevant relationship appears to be the one with the current
account. We thus have the following structure:

\startXPLAIN
type account		      = ..., case
type mortgage account = [account], ..., current account
type saving account   = [account], ...
type business account = [account], ...
type current account  = [account], ...
\stopXPLAIN

There can be little doubt about the relationship between account and
office, because each account is related to a customer and held by one
office and various accounts can be related to one customer.

One of the questions which can be considered during the design of the
account concept is whether or not name, address, town and other
specific customer data should be included in a separate data
structure. In this case study we prefer to register customers. One of
the best arguments is that banks must be able to relate debit and
credit amounts in the various accounts of any one customer. This
implies a relationship between the various accounts a customer may
have. We will introduce an external personal registration (e.g. fiscal
identification, passport number and Chamber of Commerce registration
number), and suggest regular checks for non-account holders in
addition. The following type definitions therefore apply:

\startXPLAIN
type account  = holder, ...
type holder   = ..., office, identification.
\stopXPLAIN

The final section of the example describes the relationships between
shares, bonds, deposits and accounts. Of these, deposits are not
physical items in banks - witness statements like:

\startitemize
\item deposits are linked to accounts, which suggests  an aggregation
  link between deposits and accounts;
\item multiple series of one stock in a deposit, suggesting an
  aggregation link between bonds and deposits.
\stopitemize

The noun deposits stems from the verb to deposit, which really means
"to link stocks to accounts". Deposits themselves can therefore not be
associated with properties. Thus we arrive at aggregation of stocks in
relation to accounts, where stocks consist of shares or bonds. Obvious
specializations lead us to:

\startXPLAIN
type stock  = account, ..., case
type share  = [stock], ...
type bond   = [stock], ...
\stopXPLAIN

We now have defined all relationships, and the addition of the
properties specified in the description results in the following total
model:

\startXPLAIN
type currency (I4)          = currency code, exchange_rate.
type office (I6)            = address, postalcode, town,
                              telephone number, postal_giro, bank,
                              mycase.
type head office (I6)       = [office], region.
type branch (I6)            = [office], head office.
type holder (I9)            = name, address, postalcode, town, office,
                              identification.
type account (I9)           = holder, balance, transaction_date,
                              currency, mycase.
type current account (I9)   = [account], dr_interest, cr_interest,
                              cr_limit, cheque card, Eurocheque.
type mortgage account (I9)  = [account], current account, interest,
                              redemption method, collection_indication.
type saving account (I9)    = [account], condition, interest.
type business account (I9)  = [account], dr_interest, cr_interest,
                              cr_commission, cr_limit.
type stock (I9)             = account, nominal_stockvalue,
                              purchase_stockvalue, purchase_date,
                              mycase.
type share (I9)             = [stock], number.
type bond (I9)              = [stock], start_number, last_number.
\stopXPLAIN

The above is summarized in the abstraction hierarchy in
figure~\in[fig:bank]. This figure is not entirely correct as there is
a relation from branch to head office (every branch has a head
office), and from mortgage account to current account (every mortgage
account has a current account). In the next revision of this document,
the correct figure will be presented (when I know how to do this with
MetaPost).

\placefigure
  []
  [fig:bank]
  {Bank abstraction hierarchy}
  {\startMPcode input bank.mp \stopMPcode}

\stopbodymatter


\startbackmatter

\completeindex

\chapter{References}

\placepublications[criterium=text]

\stopbackmatter

\stoptext
